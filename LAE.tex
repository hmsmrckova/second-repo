
\documentclass[12pt,english]{article}%
\usepackage{amsfonts}
\usepackage{amsmath}
\usepackage{tikz}
\usepackage{amsthm}
\usepackage[round]{natbib}
\usepackage{graphicx}
\usepackage{setspace}
\usepackage{color}
\usepackage{eurosym}
\usepackage{cancel}
\usepackage{amssymb}%
\usepackage{subfig}
\usepackage{enumerate}
\setcounter{MaxMatrixCols}{30}
\providecommand{\U}[1]{\protect\rule{.1in}{.1in}}
\renewcommand{\baselinestretch}{1.3}
\renewcommand{\arraystretch}{1.2}
\makeatletter
\renewcommand{\section}{\@startsection{section}{1}{0mm}{-1.5\baselineskip}{0.8\baselineskip}{\normalfont\large\centering}}
\renewcommand{\subsection}{\@startsection{subsection}{2}{0mm}{-0.1\baselineskip}{0.5\baselineskip}{\normalfont\bf\flushleft}}
\renewcommand{\@seccntformat}[1]{\csname the#1\endcsname
	\hspace{+0mm}\large{.}\hspace{+1.9mm}}
\renewcommand{\@seccntformat}[2]{\csname the#1\endcsname
	\hspace{+0mm}\large{.}\hspace{+1.9mm}}
\makeatother
\newtheorem{theorem}{Theorem}
\newtheorem{assumption}{Assumption}
\newtheorem{acknowledgement}{Acknowledgement}
\newtheorem{algorithm}{Algorithm}
\newtheorem{axiom}{Axiom}
\newtheorem{case}{Case}
\newtheorem{claim}{Claim}
\newtheorem{conclusion}{Conclusion}
\newtheorem{condition}{Condition}
\newtheorem{conjecture}{Conjecture}
\newtheorem{corollary}{Corollary}
\newtheorem{criterion}{Criterion}
\newtheorem{definition}{Definition}
\newtheorem{exercise}{Exercise}
\newtheorem{lemma}{Lemma}
\newtheorem{notation}{Notation}
\newtheorem{problem}{Problem}
\newtheorem{proposition}{Proposition}
\newtheorem{remark}{Remark}
\newtheorem{solution}{Solution}
\newtheorem{summary}{Summary}
\bibpunct{(}{)}{;}{a}{,}{,}
\setlength{\textwidth}{17cm} \setlength{\textheight}{22cm}
\addtolength{\oddsidemargin}{-15mm} \addtolength{\topmargin}{-5mm}
\renewcommand{\theequation}{\arabic{equation}}
\setlength{\parskip}{1mm}
\newlength{\extraspace}
\setlength{\extraspace}{.5mm}
\newlength{\extraspaces}
\setlength{\extraspaces}{2.5mm}
\def\inbar{\,\vrule height1.5ex width.4pt depth0pt}
\font\rms=cmr12 at 12pt
\def\ce{\relax\ifmmode\mathchoice
	{\hbox{$\inbar\kern-.3em{\rm C}$}} {\hbox{$\inbar\kern-.3em{\rm C}$}}
	{\lower.9pt\hbox{\rms $\inbar\kern-.3em{\rm C}$}} {\lower1.2pt\hbox{\rms
			$\inbar\kern-.3em{\rm C}$}} \else{$\inbar\kern-.3em{\rm C}$}\fi}
\font\cmss=cmss12 \font\cmsss=cmss12 at 12pt
\def\ze{\relax\ifmmode\mathchoice
	{\hbox{\cmss Z\kern-.4em Z}}{\hbox{\cmss Z\kern-.4em Z}} {\lower.9pt\hbox{\cmsss
			Z\kern-.4em Z}} {\lower1.2pt\hbox{\cmsss Z\kern-.4em Z}}\else{\cmss Z\kern-.4em Z}\fi}
\newcommand{\refsection}[1]{
	\vspace{1mm} \pagebreak[3] \addtocounter{section}{1}
	\begin{center}
		{\large #1}
	\end{center}
	\nopagebreak
	\medskip
	\nopagebreak}
\def\thebibliography#1{\refsection{\bf References}\list
	{\relax}{\itemsep=0pt \parsep=5pt
		\usecounter{enumiv}\leftmargin=3em\itemindent=-\leftmargin} \def\newblock{\hskip .11em plus .33em minus .07em}
	\sloppy\clubpenalty4000\widowpenalty4000
	\sfcode`\.=1000\relax}
\let\endthebibliography=\endlist
\newcommand{\startappendix}{
	\renewcommand{\thesection}{\Alph{section}}
	\setcounter{section}{0}
	\renewcommand{\theequation}{\thesection.\arabic{equation}}}
\newcommand{\newappendix}[1]{
	\vspace{3mm} \pagebreak[3] \addtocounter{section}{1} \setcounter{equation}{0}
	\setcounter{subsection}{0}
	\begin{center}
		{\large Appendix \thesection. #1} \vspace{0mm}
	\end{center}
	\nopagebreak \vspace{-1mm}
	\nopagebreak}
\makeatother

\newenvironment{jan}{\color{red}{ }{}}


\makeatletter
\def\@seccntformat#1{%
  \expandafter\ifx\csname c@#1\endcsname\c@section
  Exercise \thesection:
  \else
  \csname the#1\endcsname\quad
  \fi}
\makeatother


\begin{document}
	
	\title{Law and Economics written assignment}
	\author{Takumin Wang 100131\\Hana Marie Smrckova 537009} \maketitle
	
	\newpage



\section{Tort Liability}
	\begin{enumerate}[(a)]
		\item First-best precaution:
		Social Planer's Cost Minimization is
		$$x^*=\arg\min_x c_I+h=x+h(x,\upsilon)=\arg\min_x x+10\upsilon{e^{-x}}$$\textit{where $c_I$ represents costs of precaution for the injurer.}
		$$FOC: 1 -p10\upsilon{e^{-x}}=0$$
		$$x^*=ln10p\upsilon=ln5p$$ 
		\item 
		Standard of due care under a simple negligence rule should be set at $x^*=ln5p$. If the standard is set on the socially optimal level, the injurer is motivated to undertake this precaution in order to avoid liability.
				\[
        C_I=\begin{cases}
        x+h(x,v)p & ,x<x^* \\
        x & ,x\geq x^*
        \end{cases}
        \]
        \[x^*=\arg\min C_I \]
	

		\item If there are two types of injurers with different probability of accident $0<p_l<1/5<p_h<1$, the efficient levels of care $x_t\in\{l,h\}$ are $x^*=ln5p_t$. 
			$$\min_{x_h, x_l} E(c_I+h)=E(x_t+h(x_t,\upsilon))=1/2[x_h+h(x_h,\upsilon)]+1/2{x_l+h(x_l,\upsilon)}$$
			$$ FOC: x^*_t=ln5p_t , t \in \{l,h\}$$
		\item Strict liability do this work and force injurer to internalize the social costs. In other words, by setting the strict liability rule, the cost function of injurer becomes identical to the social cost function. Thus, strict liability solves the problem of public unobservability of injurer's type since injurer who knows her type privately minimize (social) costs. 
	    \\ Each type of an injurer solves:
		$$\min_{x{_t}} c_I+h=x_t+h(x_t,\upsilon)$$
			$$ FOC: x^*_t=ln5p_t \textit{ (FB)}$$ 
		\item Efficiency requires that all types of victims greater than $\bar{\upsilon}=\frac{2c}{5(p_he^{-x_h^*}+p_le^{-x_l^*})}$ take precautionary measure. We know that the following inequality must hold if precaution of victim is efficient in order to minimize social costs:\footnote{Note that it is possible to obtain the same result by minimizing social costs: $$\arg\min_{m}E[x_t+h'(x_t,\upsilon,m)p_t+mc]$$ } 
		$$\frac{1}{2}[p_h\frac{1}{2}h(\upsilon,x_h)]+\frac{1}{2}[p_l\frac{1}{2}h(\upsilon,x_l)]+c\leq\frac{1}{2}[p_hh(\upsilon,x_h)]+\frac{1}{2}[p_lh(\upsilon,x_l)]$$
		$$c\leq\frac{1}{4}p_h10\upsilon{e^{-x_h}}+\frac{1}{4}p_l10\upsilon{e^{-x_l}}$$
		$$\upsilon\geq\frac{2c}{5(p_he^{-x_h^*}+p_le^{-x_l^*})}\equiv\bar{\upsilon}$$
		$$\therefore m^*= \begin{cases}
            0 &, \upsilon<\bar{\upsilon} \\
            1 &, \upsilon\geq \bar{\upsilon}
          \end{cases}.$$
		Given the threshold derived above, the social planer's cost minimization is as follows:
		$$\arg\min_{x_h,x_l}W=E[x_hh'(x_h, \upsilon,m)p_h+mc]+E[x_lh'(x_l, \upsilon,m)p_l+mc]$$
		$$= \frac{1}{2}[x_h+p_h[prob(\upsilon\geq\bar{\upsilon})\frac{1}{2}10E[\upsilon|\upsilon\geq\bar{\upsilon}]e^{-x_h}]+ p_h[prob(\upsilon\leq\bar{\upsilon})]10E[\upsilon|\upsilon\leq\bar{\upsilon}]e^{-x_h}]$$
		$$+\frac{1}{2}[x_l+p_l[prob(\upsilon\geq\bar{\upsilon})\frac{1}{2}10E[\upsilon|\upsilon\geq\bar{\upsilon}]e^{-x_l}]+ p_l[prob(\upsilon\leq\bar{\upsilon})]10E[\upsilon|\upsilon\leq\bar{\upsilon}]e^{-x_l}]+c-\bar{\upsilon}c$$
		$$=p_h\frac{1}{2}((\frac{5}{2}+\frac{5}{2}\bar{\upsilon^2})e^{-x_h})+p_l\frac{1}{2}((\frac{5}{2}+\frac{5}{2}\bar{\upsilon^2})e^{-x_l})+\frac{1}{2}(x_h+x_l)-c\bar{\upsilon}+c$$
		Thus we have the first-order conditions for maximization of social welfare:
		$$FOC_{x_h}:-\frac{1}{2}p_h((\frac{5}{2}+\frac{5}{2}\bar{\upsilon^2})e^{-x_h^*})+\frac{1}{2}=0$$
		$$FOC_{x_l}:-\frac{1}{2}p_l((\frac{5}{2}+\frac{5}{2}\bar{\upsilon^2})e^{-x_l^*})+\frac{1}{2}=0$$
   					
		\item Neither strict liability rule with contributory negligence nor standard negligence rule can implement the first-best levels of care $x_t^*$ and $m=1$ for $v\geq\bar{\upsilon}$. In the strict liability rule with contributory negligence, it is not possible to design a standard of due precaution of victims that brings efficient level of precaution. Victims do not take optimal level of precaution since their type is not observable and they can pretend to be of the low $\upsilon$ types in case the rule is to take precaution if $\bar{\upsilon}\leq\upsilon$. If the standard due care for all victims is set to invest ($m=1$), at least some types of victims will take inefficient precaution in order to avoid liability. In standard negligence rule, high risk injurer is motivated to imitate low risk injurer and thus take lower than optimal level of care. 
		\item The total liablity of participants to an accident equals to the total harm done if $D_v+D_I=1$ (\textit{where $D_I,D_v$ are the proportions of total harm borne by the injurer and victim, respectively.})	In order to achieve first-best we set $D_v=1$ and $D_I=0$ if $x\geq{x_h^*}$and $D_v=1-\bar{D},D_I=\bar{D_I}$ if $x<{x_h*}$. The intuition for this is that if the court would observe higher level of prevention than is the optimal due care for high type, then it is certain that the level of care was high enough for both types (in fact excessive care for the low type). On the other hand, we design a damage $\bar{D_I}$ such that the low type still would like to pay but that the high type would rather take higher precaution since the expected damage payment exceeds the saving cost on precaution. Therefore, the question of interest is how to set the $\bar{D_I}$ in order to force injurers to reveal their type (it is not profitable for high type to pretend to be low type and -- at the same time -- low type is not forced to take higher than optimal level of care). We need to find $\bar{D_I}$ such that high type wants to set $x_h^*$  and low type wants to set $x_l^*$:
		
		(i) high type wants to reveal himself:
		$${x_h^*}\leq{x}+D_Ih'(x_l^*,\upsilon,m)p_h$$
		$${x_h^*}\leq{x_l^*}+D_I[\{(1-\bar{\upsilon})5\frac{1+\bar{\upsilon}}{2}+\bar{\upsilon}10\frac{\bar{\upsilon}}{2}\}e^{-x_l^*}]p_h$$
		$${x_h^*}\leq{x_l^*}+D_I(\frac{5}{2}+\frac{5}{2}\bar{\upsilon^2})e^{-x_l^*}p_h$$
		Note that when the injurer is liable to $\bar{D_I}$ time the total harm, it is always optimal for her to set $x_l^*$, so   condition (i) is sufficient for high type to reveal himself. \\
		(ii) low type does not want to pretend to be high
		$${x_l^*}+D_Ih'(x_l,\upsilon,m)p_l\leq{x_h^*}$$
		$${x_l^*}+D_I(\frac{5}{2}+\frac{5}{2}\bar{\upsilon^2})e^{-x_l}p_l\leq{x_h^*}$$
		(iii) low type prefers $x_l^*$ to $x$
		$${x_l^*}+D_Ih'(x_l,\upsilon,m)p_l\leq{x}+D_Ih'(x,\upsilon,m)p_l$$
	    Note that  (iii) is redundant, above conditions can be reduced to:
		$$\frac{x_h^*-x^*_l}{(\frac{5}{2}+\frac{5}{2}\bar{\upsilon^2})e^{-x^*_l}p_h}\leq{\bar{D_I}}\leq\frac{x_h^*-x_l^*}{(\frac{5}{2}+\frac{5}{2}\bar{\upsilon^2})e^{-x_l^*}p_l}$$
		This condition is always satisfied since $p_h>p_l$, so such $\bar{D_I}$ always exists.\\
		In sum, the high type of injurers set $x_h^*$ to exempt from liability of damage, and the low type chooses $x_l^*$  to save costs and bears risk of being liable to $\bar(D_I)$ time the harm. Knowing this, the higher type of victims $\upsilon\geq\bar{\upsilon}$ thus take precaution according to the argument in (e), so we can implement the first best level of care. 
		\\
		\\The liability rule involve negative damages if $\bar{D_I}\geq1\Leftrightarrow{\bar{D_v}\leq0}$. The condition for negative damages is as follows: 
		$$\frac{x_h^*-x_l^*}{(\frac{5}{2}+\frac{5}{2}\bar{\upsilon^2})e^{-x^*_l}p_h}\geq1.$$
		In this case, victims obtain higher compensation than is the harm she suffered. Intuitively, when the risk of harming is sufficiently low, negative damages serve to discourage the high type injurer from pretending to be low type since he would face punitive damages. Low type is willing to bear punitive damages to reveal himself because she bears lower risk than high type and would rather save some precaution cost. Under this liability rule, victims know that each type of injurers will take optimal precaution of high type, so type  $\upsilon\geq\bar{\upsilon}$ is willing to undertake precaution.
		\item To reveal all the specific details of the case is costly. For example, in some cases it is more efficient to set the liability rule that makes injurers internalize social costs and give the optimal outcome in expectation rather than to determine for instance the standard of due care on the case-to-case basis, which may be costly to evaluate. To illustrate, it is more efficient to set an uniform speed limit on the roads rather than set individual speed limit based on the driving skills of each driver. In fact, the limit should be set on the level when marginal social loss from opportunity time costs of more skilled drivers is equal to marginal saving on revealing individual driving skills in case of accident occurrence.
		
		Furthermore, from question (g) we see that even if the courts do not observe types but only precaution level, the court still can decouple damage from harm and design a direct mechanism to elicit information (truthful revelation) and thus provide right incentives.
		
	\end{enumerate}
	
\section{Marginal Deterrence}	
	Denote $a_o$ such that type $t_o$ choose $a_o$ in the optimal punishment scheme.
	First, can social planner implement optimal $a_o^*$ such that to $b'(a_o^*)=h'(a_o^*)$? The answer is "No", because  monitoring comes at costs, and it could be socially well off to allow for some harmful activity to save the monitoring cost. Since $c_p=0$, we have $p(a)=1$, $e(a)=\mu{f(a)}$ and the social welfare function is:
	$$W=t_0b(a_o)-h(a_o)-\mu{c_m}$$ 
	To implement $a_o$ such that any ${a}>a_0$ is deterred, 
	 $t_0$ chooses $a_0$ that maximizes her individual utility $u(a)=t_0b(a)-e(a)$. Therefore, for any ${a}>a_0$, we have
	 $$t_0b(a)-e(a)\leq{t_0b(a_0)}-e(a_0).$$
	 Note that since we do not deter $a\leq{a_0}$, we can set $f(a)=0$ for all ${a}{\leq}{a_0}$ and thus $\Rightarrow{e(a_0)=0}$. Hence, we can set $e(a)$ at minimum $t_0b(a)-t_0b(a_0)$ to minimize monitoring cost.
	 \\Taking $a\rightarrow\infty$ and setting highest sanction to highest harm, we have
	 \begin{equation}
	 \mu{w}=\lim_{a\rightarrow\infty}e(a)=t_0\bar{b}-t_0b(a_0)
	\end{equation}
	Thus following condition must hold for $t_0$ to choose less harmful acts:
	  $$w\geq\mu{w}=t_0\bar{b}-t_0b(a_0)$$
	 \begin{equation}t_0b(a_0)\geq{t_0\bar{b}}-w\end{equation} If equation (2) is not satisfied, then no deterrence schedules can be implemented. We substitute equation (1) in social welfare function and $\frac{v}{w}$ as a Lagrange multiplier on the constraint:
	 $$\max{t_0}b(a_0)-h(a_0)-\frac{c_M}{w}(t_0\bar{b}-t_0b(a_0))$$
	 $$\textit{s.t. }w\geq{t_0}\bar{b}-t_0b(a_0)$$
	 \begin{equation}
	FOC:[t_0+\frac{c_M+v}{w}]b'(a_0)=h'(a_0)
	 \end{equation}
	 Since $c_M>0$ $v\geq0$, $a_0\geq{a_0^*}$. If $v>0$ then monitoring cost is at its maximum $\mu=1$. For $v=0$ then $\mu<1$. Only when $\frac{w}{C_M}\rightarrow\infty$ can $a_0=a_0^*$ be achieved.\\
	 To summarize, the optimal sanction scheme is: 
	 	\[
        f(a)=\begin{cases}
        t_0b(a)-t_0b(a_0) & ,a>a_0 \\
        0 & ,a\leq{a_0} 
        \end{cases}
        \]
        ,where $a_0$ is characterized by (3) and $\mu$ is characterized by (1).

\section{Litigation}
	 Osborne,(1999): Who should be worried about asymmetric information in litigation?
	 \paragraph{Research Question:} Is litigation process characterized by asymmetric information or by random optimism of the litigants? Who suffers by asymmetric information?  
	 \paragraph{Relevance:}It tries to resolve the question that is important for the optimal setting of the litigation rules.
	 \paragraph{Method:}Empirical analysis. The econometric model based on cross-sectional data examines how the final award to the plaintiff from the litigation is determined by the attorney's estimate of the maximum the client should have taken to settle the case.   
	 \paragraph{Answer:}Asymmetric information are present. Defendants are significantly better in estimation of the outcome. This seems to be especially the case for the plaintiffs who pay their attorneys contingency fee. The possible explanation is that under contingency fee, attorney is motivated to acquire lower than optimal evidence. 
	 \paragraph{Evluation:}We see the approach of this paper rather superficial. Problem of this analysis arises even in the dataset. The crucial explanatory variable is supposed to catch \textit{ex ante} estimations of the attorney who are, however, asked to give this estimate after the case is resolved and the outcome is known. Thus, the estimates can be severely affected by hindsight bias. Moreover, the author does not discuss the possible endogeneity that is likely to be present in the model. The most important objection is that there is no theoretical framework in this paper that fits its methodology. In particular, the author refers to the paper written by Waldfogel (1998) who also asked whether asymmetric information or random optimism (divergent expectations) characterizes litigation process. Unlikely to Osbourne, however, Waldfogel use the probability of win as a proxy. We do not think that to be optimistic about the winning probability is the same as to be optimistic about the amount awarded. Thus, it would be valuable if the author formulates the theoretical framework more precisely and if he explains why his approach is supposed to deliver more reliable results than Wladfogel's. Moreover, the sample size is quite small in comparison to Wladfogel's.	 	 
	   



\end{document}
